\documentclass{beamer}

\usepackage[utf8]{inputenc}% -> pour les accents
\usepackage[T1]{fontenc}
%\usepackage[a4paper,left=3cm,right=3cm,top=3cm,bottom=3cm]{geometry}
\usepackage[english]{babel} % -> en francais
\usepackage{amsmath} % -> symboles mathématiques. amsmath inclut d'autres packages pour écrire en math comme amssymb,amstext, ...
\usepackage{hhline}% -> package supplémentaire permettant le double soulignement
\usepackage{amsfonts}
\usepackage{amssymb}
\usepackage{graphicx} % -> pour pouvoir mettre des dessins

\usepackage[toc,page]{appendix}
\usepackage{url}
\usepackage{color}
\usepackage{shadow}
\usepackage{fancybox}
\usepackage[babel=true]{csquotes}
\usepackage{bigfoot} % pour utiliser \verb dans une footnote
\usepackage{bm}

\def\N{\mathbb{N}}
\def\R{\mathbb{R}}
\def\Q{\mathbb{Q}}
\def\Z{\mathbb{Z}}
\def\C{\mathbb{C}}

\author{Cédric Simal}
\date{\today}
\title{Do you believe in Real numbers?}


\usetheme{Boxes}

\setbeamertemplate{navigation symbols}{
	\hspace{1em} \usebeamerfont{footline} \insertframenumber/\inserttotalframenumber }

\begin{document}

\frame{\titlepage}

\begin{frame}
\frametitle{Motivation}


\end{frame}

\begin{frame}
\frametitle{Philosophical Hazards}
\framesubtitle{Side effects of this talk may include}
\begin{itemize}[<+->]
    \item Indifference
    \item Mild interest
    \item Five stages of grief
    \item Constructivism
\end{itemize}

\end{frame}

\begin{frame}
    \frametitle{Today's menu}
    \begin{enumerate}
        \item The arithmetic definition of real numbers
        \item Computable numbers
        \item Algorithmically random numbers
    \end{enumerate}
\end{frame}

\begin{frame}
    \frametitle{Peano Axioms (1889)}
    \begin{columns}
        \begin{column}{.75\textwidth}
            \begin{enumerate}
                \item 0 is a natural number
                \item If $n$ is a natural number, its successor $S(n)$ is a natural number
                \item For all $n$, $S(n) \neq 0$
                \item For all $m$, $n$, $S(m) \neq S(n) \Rightarrow m \neq n$
            \end{enumerate}
        \end{column}
        \begin{column}{.25\textwidth}
            \includegraphics[height=.25\textheight]{images/Giuseppe_Peano.jpg}
            \vfill
        \end{column}
    \end{columns}
    \vspace{2.5em}
    \pause[2]
    We call $\N = \{ 0, S(0), S(S(0)), \dots \}$ the set of natural numbers.
\end{frame}

\begin{frame}
    \frametitle{Arithmetic on Peano numbers}
    \framesubtitle{Addition and multiplication are defined recursively}
    For all $m$, $n \in \N$
    \begin{itemize}
        \item $m+0 = m$, and $m+S(n) = S(m+n)$
        \item $m * 0 = 0$, and $m * S(n) = m*n + m$
    \end{itemize}
\end{frame}

\begin{frame}
    \frametitle{Integers}
    $$\Z = \{ (m,n) ~|~m,n \in \N \}, $$ 
    where $(m,n)$ is interpreted as $m-n$.

    \vspace{2.5em}
    Relations on $\Z$:
    \begin{itemize}
        \item $(m_1,n_1) = (m_2,n_2) ~\Leftrightarrow m_1 + n_2 = m_2 + n_1$
        \item $(m_1,n_1) < (m_2,n_2) ~\Leftrightarrow m_1 + n_2 < m_2 + n_1$
    \end{itemize}
\end{frame}

\begin{frame}
    \frametitle{Arithmetic on Integers}
    \begin{itemize}
        \item $(m_1,n_1) + (m_2,n_2) = (m_1+m_2, n_1, n_2) $
        \item $(m_1,n_1) * (m_2,n_2) = (m_1*m_2 + n_1*n_2, m_1*n_2 + m_2*n_1)$
        \item $-(m,n) = (n,m)$ (Additive inverse)
    \end{itemize}

    \pause[2]
    \vspace{2.5em}
    $(\Z, +, *)$ is a \textit{ring}.
\end{frame}

\begin{frame}
    \frametitle{Rationals}
    $$ \Q = \{ (p,q)~| p,q \in \Z, q > 0 \}, $$
    where $(p,q)$ is interpreted as $\frac{p}{q}$.

    \vspace{2.5em}
    Relations on $\Q$
    \begin{itemize}
        \item $(p_1, q_1) = (p_2,q_2) \Leftrightarrow p_1 * q_2 = p_2 * q_1$
        \item $(p_1, q_1) < (p_2,q_2) \Leftrightarrow p_1 * q_2 < p_2 * q_1$
    \end{itemize}
\end{frame}

\begin{frame}
    \frametitle{Arithmetic on Rationals}
    \begin{itemize}
        \item $(p_1,q_1) + (p_2,q_2) = (p_1*q_2 + p_2*q_1, q_1*q_2)$
        \item $(p_1,q_1) * (p_2,q_2) = (p_1*p_2,q_1*q_2)$
        \item $(p,q)^{-1} = (q,p)$ (Multiplicative inverse)
    \end{itemize}

    \pause[2]
    \vspace{2.5em}
    $(\Q, +, *)$ is a \textit{field}.
\end{frame}

\begin{frame}
    \frametitle{Dedekind cuts}
\end{frame}

\begin{frame}
    \frametitle{Interlude}
    \framesubtitle{Fun problems if you're bored}
    \begin{itemize}
        \item Explicitly define a bijection between $\mathcal{P}(\N)$ and $[0,+\infty)$.
        \item Show that the set of continuous functions from $\R$ to $\R$ has the same cardinality as $\R$
    \end{itemize}
\end{frame}

\begin{frame}
    \frametitle{Definable numbers}
\end{frame}

\begin{frame}
    \frametitle{Computable numbers}
\end{frame}

\begin{frame}
    \frametitle{}
\end{frame}

\begin{frame}
    \frametitle{That's all folks}
    Slides and references can be found at
    \url{https://github.com/csimal/are-real-numbers-real}
\end{frame}

\end{document}